%%
%% This is file `sample-acmsmall.tex',
%% generated with the docstrip utility.
%%
%% The original source files were:
%%
%% samples.dtx  (with options: `acmsmall')
%% 
%% IMPORTANT NOTICE:
%% 
%% For the copyright see the source file.
%% 
%% Any modified versions of this file must be renamed
%% with new filenames distinct from sample-acmsmall.tex.
%% 
%% For distribution of the original source see the terms
%% for copying and modification in the file samples.dtx.
%% 
%% This generated file may be distributed as long as the
%% original source files, as listed above, are part of the
%% same distribution. (The sources need not necessarily be
%% in the same archive or directory.)
%%
%% The first command in your LaTeX source must be the \documentclass command.
\documentclass[acmsmall]{acmart}

%%
%% \BibTeX command to typeset BibTeX logo in the docs
\AtBeginDocument{%
  \providecommand\BibTeX{{%
    \normalfont B\kern-0.5em{\scshape i\kern-0.25em b}\kern-0.8em\TeX}}}

%% Rights management information.  This information is sent to you
%% when you complete the rights form.  These commands have SAMPLE
%% values in them; it is your responsibility as an author to replace
%% the commands and values with those provided to you when you
%% complete the rights form.


%%
%% These commands are for a JOURNAL article.

%%
%% Submission ID.
%% Use this when submitting an article to a sponsored event. You'll
%% receive a unique submission ID from the organizers
%% of the event, and this ID should be used as the parameter to this command.
%%\acmSubmissionID{123-A56-BU3}

%%
%% The majority of ACM publications use numbered citations and
%% references.  The command \citestyle{authoryear} switches to the
%% "author year" style.
%%
%% If you are preparing content for an event
%% sponsored by ACM SIGGRAPH, you must use the "author year" style of
%% citations and references.
%% Uncommenting
%% the next command will enable that style.
%%\citestyle{acmauthoryear}

%%
%% end of the preamble, start of the body of the document source.
\begin{document}

%%
%% The "title" command has an optional parameter,
%% allowing the author to define a "short title" to be used in page headers.
\title{Tarea No. 4. Procesamiento de Grafos}

%%
%% The "author" command and its associated commands are used to define
%% the authors and their affiliations.
%% Of note is the shared affiliation of the first two authors, and the
%% "authornote" and "authornotemark" commands
%% used to denote shared contribution to the research.
\author{Luis Daniel Roa González}
\email{a01021960@itesm.mx}
\orcid{A01021960}
\affiliation{%
  \institution{Instituto Tecnológico de Estudios Superiores de Monterrey, Campus Santa Fe}
  \streetaddress{Av Carlos Lazo 100, Santa Fe, La Loma}
  \city{Santa Fe}
  \state{Ciudad de México}
  \postcode{01389}
}

%%
%% By default, the full list of authors will be used in the page
%% headers. Often, this list is too long, and will overlap
%% other information printed in the page headers. This command allows
%% the author to define a more concise list
%% of authors' names for this purpose.
%%
%% The abstract is a short summary of the work to be presented in the
%% article.
\begin{abstract}
  El motivo de esta tarea es, poder identificar que tipo de algoritmo para convertir un documento de texto a diferentes tipos de grafos es el mas eficiente, es decir, cual es el mas rápido, el mas eficiente y tanto las ventajas como las desventajas de cada uno de los formatos utilizados. 
  
  Como meta adicional, se aprendió a utilizar el formato oficial de la ACM (Association for Computing Machinery) y se llevó a cabo utilizando \LaTeX\ .
\end{abstract}
%%
%% The code below is generated by the tool at http://dl.acm.org/ccs.cfm.
%% Please copy and paste the code instead of the example below.
%%


%%
%% Keywords. The author(s) should pick words that accurately describe
%% the work being presented. Separate the keywords with commas.
\keywords{grafos, tarea, analisis, diseño, algoritmos, ITESM, \LaTeX\ }


%%
%% This command processes the author and affiliation and title
%% information and builds the first part of the formatted document.
\maketitle

\section{Introducción}
Esta tarea tuvo como finalidad, poder convertir un \textit{dataset} que escogieramos de la página de Snap, la cual ofrece diferentes tipos de \textit{datasets} para que se pueda conllevar una simulación acerca del funcionamiento de la red elegida.

Para el propósito de esta investigación, se llevó a cabo usando el dataset llamado \textbf{wiki-Vote}, el cual consiste en la cantidad de votos que se llevaron a cabo para poder aceptar a miembros nuevos como administradores para que puedan editar el sitio a su gusto. Este dataset contiene información desde el momento en el cual Wikipedia surgió (2001) hasta la fecha en la que se recolectó esta información (2008).

Al llevar a cabo esta tarea, se descargó un programa para poder abrir documentos con formatos de grafos, este programa se llama \textbf{Gephi}, este mismo permite visualizar el sistema de grafos.

Para conllevar de manera correcta esta tarea, fue requerido realizar un programa en C++ que pudiera convertir el documento de texto con la información al respecto del grafo, a los siguientes formatos:
\begin{itemize}
\item GraphML
\item GEXF
\item GDF
\item JSON
\end{itemize}


\section{Programa realizado}
Como se mencionó en la introducción, se realizó un programa donde se inserta el nombre del documento de texto (\textit{con terminación .txt}) el cual contiene el nodo principal y el nodo al que se encuentra conectado. El programa está compuesto usando dos diferentes archivos, un \textit{main} con terminación \textbf{.cpp}, este lleva el nombre de \textit{Tarea4.cpp}, la única funcionalidad de este programa es medir los tiempos y mandar a llamar los métodos que se están utilizando para la creación de los documentos que contendrán los grafos.

El otro programa del que se menciona, es un programa tipo \textit{header}, en este se encuentran los métodos donde se convierte el documento de texto a los formatos que se especificaron en la introducción de este documento. Cada método corre después de que el anterior finalizó.


\subsection{Conversión de \textit{.txt} a \textit{GrapML}}
La manera 

\subsection{Conversión de \textit{.txt} a \textit{GEXF}}

\subsection{Conversión de \textit{.txt} a \textit{GDF}}

\subsection{Conversión de \textit{.txt} a \textit{JSON}}

\section{Gefi y sus funcionalidades}

\section{Conclusión}

\section{Referencias}
\subsection{Código}
El código fue obtenido de:

\subsection{Gefi}
Gefi fue obtenido de:

\subsection{LaTeX}
La distribución de \LaTeX\ no fue descargada a la computadora, por lo tanto se uso la distribución encontrada en línea llamada \url{https://www.overleaf.com}.

\end{document}
\endinput
%%
%% End of file `sample-acmsmall.tex'.
